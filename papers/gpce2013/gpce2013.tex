\documentclass[preprint]{sigplanconf}

% The following \documentclass options may be useful:
%
% 10pt          To set in 10-point type instead of 9-point.
% 11pt          To set in 11-point type instead of 9-point.
% authoryear    To obtain author/year citation style instead of numeric.

\usepackage{amsmath}
\usepackage[T1]{fontenc}
\usepackage[utf8]{inputenc}
\usepackage[unicode=true,hidelinks]{hyperref}
\usepackage{graphicx}

\usepackage{hyperref}
\usepackage{listings}
%\usepackage[scaled]{luximono}

%\usepackage{natbib}

% ----- begin macros

\lstdefinelanguage{Scala}%
{morekeywords={abstract,%
  case,catch,char,class,%
  def,else,extends,final,for,%
  if,import,implicit,%
  match,module,%
  new,null,%
  object,override,%
  package,private,protected,public,%
  for,public,return,super,%
  this,throw,trait,try,type,%
  val,var,%
  with,%
  yield,%
  lazy%
  },%
  sensitive,%
  morecomment=[l]//,%
  morecomment=[s]{/*}{*/},%
  morestring=[b]",%
  morestring=[b]',%
  showstringspaces=false%
}[keywords,comments,strings]%

\lstdefinelanguage{JavaScript}%
{morekeywords={for, var, attributes, class, classend, do, else, empty, endif, endwhile, fail, function,
functionend, if, implements, in, inherit, inout, not, of, operations, out, return,
then, types, while, use},%
  sensitive,%
  morecomment=[l]//,%
  morecomment=[s]{/*}{*/},%
  morestring=[b]",%
  morestring=[b]',%
  showstringspaces=false%
}[keywords,comments,strings]%

\lstset{language=Scala,%
  mathescape=false,%
%  columns=[c]fixed,%
  aboveskip=\smallskipamount,
  belowskip=\smallskipamount,
%  basewidth={0.5em, 0.4em},%
  basicstyle=\footnotesize,%
  captionpos=b,
  keywordstyle=\bfseries%\sffamily\bfseries,%
%  keywordstyle=\sffamily\bfseries,%
%  xleftmargin=0.5cm
}

\newcommand{\commentstyle}[1]{\slseries{#1}}
\newcommand{\keywordstyle}[1]{\bfseries{#1}}

\lstnewenvironment{slisting}{\lstset{language=Scala}}{}

\newcommand{\code}[1]{\lstinline[language=Scala,columns=fixed,basicstyle=\footnotesize]|#1|}

\def\changemargin#1#2{\list{}{\rightmargin#2\leftmargin#1}\item[]}
\let\endchangemargin=\endlist

%\setlength{\columnseprule}{0.25pt}

%\renewcommand{\note}[1]{$\spadesuit$ \textbf{#1} $\clubsuit$}


%\newcommand{\comment}[1]{}


\newcommand{\ie}{\emph{i.e.}}
\newcommand{\eg}{\emph{e.g.}}
\newcommand{\cf}{\emph{cf.~}}
\newcommand{\etal}{\emph{et al.~}}
\newcommand{\etc}{\emph{etc.}}
\newcommand{\aka}{\emph{a.k.a.}}


\begin{document}

\conferenceinfo{GPCE '13}{October 27--28, 2013, Indianapolis, Indiana, USA} 
\copyrightyear{2013}
\copyrightdata{[to be supplied]} 

\titlebanner{banner above paper title}        % These are ignored unless
\preprintfooter{short description of paper}   % 'preprint' option specified.

\title{Efficient High-Level Abstractions for Web Programming}
%\subtitle{You don’t have to trade abstraction for control}

\authorinfo{Julien Richard-Foy\and Olivier Barais\and Jean-Marc Jézéquel}
           {IRISA, Université de Rennes 1}
           {\{first\}.\{last\}@irisa.fr}

\maketitle

\begin{abstract}
Writing large Web applications is known to be difficult. One challenge comes from the fact that the application's
logic is scattered into heterogeneous clients and servers, making it difficult to share code between both sides or to
move code from one side to the other. Another challenge is performance: while Web applications rely on ever more
code on the client-side, they may run on smart phones with little hardware capabilities. These two challenges raise
the following problem: how to benefit from high-level languages and libraries making code complexity easier to manage
and abstracting over the clients and servers differences without trading this engineering comfort for performance?
This article presents high-level abstractions defined as deep embedded DSLs in Scala, that can (1) generate efficient
code leveraging the target platform characteristics, (2) be shared between clients and servers. Though code written
with our DSL has a high level of abstraction our benchmark on a real world application reports that it runs as fast
as hand tuned low-level JavaScript code.
\end{abstract}

\category{D.3.3}{Programming Languages}{Language Constructs and Features}

\terms Languages, Software Engineering

\keywords Heterogeneous code generation, Domain-specific languages, Scala, Web

\section{Introduction}

Web applications are attractive because they require no installation or deployment steps on clients and enable large
scale collaborative experiences. However, writing large Web applications is known to be
difficult~\cite{Mikkonen08_SpaghettiJs,Preciado05_RIAMethodologyNecessity}. One challenge comes from the fact
that the business logic is scattered into heterogeneous client-side and server-side
environments~\cite{Echeverria09_RIA,Kuuskeri09_PartitioningClientServer}. This gives less flexibility in the
engineering process and requires a higher maintenance effort: there is no way to move a piece of code targeting
the server-side to target the client-side, the code has to be rewritten. Even worse, logic parts that run on both
client-side and server-side need to be duplicated. For instance, HTML fragments may be built from the server-side
when a page is requested by a client, but they may also be built from the client-side to perform an incremental
update subsequent to an user action. How could developers write HTML fragment definitions once and render them on
both client-side and server-side?

The more interactive the application is, the more logic needs to be duplicated between the server-side and the
client-side, and the higher is the complexity of the client-side code. Developers use libraries and frameworks to get
high-level abstractions on client-side, making their code easier to reason about and to maintain, but also making
their code run less efficiently (due to \emph{abstraction penalty}).

Performance is a primary concern in Web applications, because they are expected to run on a broad range of devices,
from the powerful desktop personal computer to the less powerful smart phone. “Every 100~ms delay costs 1\% of
sales”, said Amazon in 2006.

Using the same programming language on both server-side and client-side could improve the software engineering
process by enabling code reuse between both sides. Incidentally, the JavaScript language -- which is currently the
most supported the action language on Web clients -- can be used on server-side, and an increasing number of
programming languages or compiler back-ends can generate JavaScript code (\eg Java/GWT~\cite{Chaganti07_GWT},
SharpKit\footnote{\href{http://sharpkit.net}{http://sharpkit.net}}, Dart~\cite{Griffith11_Dart},
Kotlin\footnote{\href{http://kotlin.jetbrains.org/}{http://kotlin.jetbrains.org/}},
ClojureScript~\cite{McGranaghan11_ClojureScript}, Fay\footnote{\href{http://fay-lang.org/}{http://fay-lang.org/}},
Haxe~\cite{Cannasse08_HaXe} or Opa\footnote{\href{http://opalang.org/}{http://opalang.org/}}).

However, using the same programming language is not enough because the client and server programming environments
are not the same. For instance, DOM fragments can be defined on client-side using the standard DOM API, but this
API does not exist on server-side. How to define a common vocabulary for such concepts? And how to make the
executable code leverage the native APIs, when possible, for performance reasons?

Generating efficient code for heterogeneous platforms is hard to achieve in an extensible way: the translation of
common abstractions like collections into their native counterpart (JavaScript arrays on client-side and standard
library's collections on server-side) may be hard-coded in the compiler, but that approach would not scale to handle
all the abstractions a complete application may use (\eg HTML fragment definitions, form validation rules, or even
some business data type that may be represented differently).

On one hand, for engineering reasons, developers want to write Web applications using a single high-level language,
abstracting over the target platforms differences and reducing code complexity. But on the other hand, for
performance reasons, they want to keep control on the way their code is compiled to each target platform. How to
solve this dilemma?

Compiled domain-specific embedded languages~\cite{Elliott2003_Compiling} allow the definition of domain-specific
languages (DSLs) as libraries on top of a host language, and to compile them to a target platform. The deep embedding
gives the opportunity to control the code generation scheme for a given abstraction and target platform.

Kossakowski \etal introduced \emph{js-scala}, a compiled embedded DSL defined in Scala that generates JavaScript
code, making it possible to write the client-side code of Web applications using
JavaScript~\cite{Kossakowski12_JsDESL}. However, Kossakowski \etal did not address the engineering dilemma described
above. This paper enriches js-scala with the following contributions\footnote{The code is available at
\href{http://github.com/js-scala}{http://github.com/js-scala}}:

\begin{itemize}
 \item we define high-level abstractions, typically used in Web programming, that generate low-level code leveraging
native Web browsers APIs;
 \item in the case of abstractions shared between servers and clients, we specialize the code generation in order to
leverage the target platform native environments.
\end{itemize}

We validate our approach with a case study implemented with various candidate technologies and discuss the relative
pro and cons of them. Though the code written in our DSL is high-level and can be shared between clients and servers,
it has the same runtime performances on client-side as hand-tuned low-level JavaScript code.

The remainder of this paper is organized as follows. The next section introduces existing approaches for defining
cross-compiling languages and presents the framework we used to define our DSLs. Sections \ref{contribution} and
\ref{contribution-shared} present our contribution. Section \ref{validation} compares our solution to common
approaches. Section \ref{discussion} concludes.

\section{Background}

\subsection{Language Engineering Processes}

This section presents different approaches for defining cross-platform programming languages.

\begin{figure}
\begin{center}
\includegraphics[width=6cm]{langs.pdf}
\end{center}
\caption{Language engineering processes}
\label{langs}
\end{figure}

\paragraph{Fat Languages}

The first approach for defining a cross-platform language consists in hard-coding, in the compiler, the
code generation scheme of each language feature to each target platform. Figure \ref{langs} (a) depicts this
process. In order to support a feature related to a specific domain, the whole compiler pipeline (parser, code
generator, \etc) may have to be adapted. This approach gives \emph{fat} languages because a lot of concepts are
defined at the language level: general programming concepts such as naming, functions, classes, as well as more
domain-specific concepts such as HTML fragment definition. Examples of such languages for Web programming are
Links~\cite{Cooper07_Links}, Opa and Dart~\cite{Griffith11_Dart}. These languages are difficult to extend because
each concept is defined in the compiler, and modifying a compiler requires a high effort. Furthermore, these
languages also require to support common programming abstraction and composition mechanisms, as general purpose
languages do. So they usually try to re-invent the features of general purpose languages, that’s why we argue that
this approach for defining programming languages is difficult to scale: for every specific problem you would have to
rewrite a full-featured programming language besides addressing the concepts specific to the problem domain.

\paragraph{Domain-Specific Languages}

Another approach consists in defining several independent domain-specific languages~\cite{Van00_DSL}, each one
focusing on concerns specific to a given problem domain, and then to combine all the source artifacts written with
these language into one executable program, as shown in figure~\ref{langs} (b). Defining such languages requires a
minimal effort compared to the previous approach because each language has a limited set of features. On the other
hand, it is difficult to have interoperability between DSLs (missing ref). \cite{Groenewegen08_WebDSL} gave an
example of such a domain-specific language for defining Web applications.

\paragraph{Thin Languages}

Alternatively, one can define concepts relative to a specific domain as a library on top of a thin general purpose
language (it is also referred to as a domain-specific \emph{embedded} language~\cite{Hudak96_DSEL}). Figure
\ref{langs} (c) depicts this approach. The general purpose language is used as a host language and does not need to
be modified if a new concept is introduced, because concepts are defined as pure libraries. On the other hand, the
syntax of the DSLs is limited by the syntax flexibility of the host language. Furthermore, this approach gives no
opportunity to translate a concept efficiently according to the target platform characteristics because concepts are
defined as libraries and are translated by the compiler, which has no domain-specific knowledge. Examples of
languages following this approach are Java/GWT, Kotlin, HaXe and SharpKit.

\paragraph{Deeply Embedded Languages}

The last approach, shown in figure \ref{langs} (d), can be seen as a middle-ground between the two previous
approaches: DSLs are embedded in a host language but use a code generation process. This approach share the same
benefits and limits as embedded DSLs for defining language units. The code generation process is specific to each DSL
and gives the opportunity to perform domain-specific optimizations. In other words deeply embedded DSLs bring
domain-specific knowledge to the compiler.

(Summary explaining why we choose the deep embedded DSLs approach)

\subsection{Lightweight Modular Staging}

Lightweight Modular Staging~\cite{Rompf12_LMSThesis, Rompf12_LMS} (LMS) is a framework for defining deeply embedded
DSLs in Scala. It has been used to define high-performance DSLs for parallel computing~\cite{Brown11_Parallel} and to
define JavaScript as an embedded DSL in Scala~\cite{Kossakowski12_JsDESL}.

LMS is based on staging~\cite{Jorring1986_Staging}: a program using LMS is a regular Scala program that evaluates
to an intermediate representation (IR) of a final program. This IR is a graph of expressions that can be traversed
by code generators to produce the final program code. Expressions evaluated in the initial program and those
evaluated in the final program (namely, staged expressions) are distinguished by their type: a \code{Rep[Int]} value
in the initial program is a staged expression that generates code evaluating to an \code{Int} value in the final
program. An \code{Int} computation in the initial program is evaluated during the initial program evaluation and
becomes a constant in the final program.

Defining a DSL with LMS consists in the following steps:

\begin{itemize}
 \item writing a Scala module providing the DSL vocabulary as an abstract API,
 \item implementing the API in terms of IR nodes,
 \item defining a code generator visiting IR nodes and generating the corresponding code.
\end{itemize}

\section{High-Level Abstraction for Client-Side Code: Selectors}
\label{contribution}

In a Web application, the user interface is defined by a HTML document that can be updated by the JavaScript code.
A typical operation consists in searching some “interesting” element in the document, in order to extract its
content, replace it or listen to user events triggered on it (such as mouse clicks). The standard API provides
several functions to search elements in a HTML document according to their name or attribute values.
Figure~\ref{selectors-api} summarizes the available functions and their differences.

\begin{figure}
\begin{center}
\begin{tabular}{| l | p{3cm} |}
\hline
Function & Description \\
\hline
\code{querySelector(s)} & First element matching the CSS selector \code{s} \\
\hline
\code{getElementById(i)} & Element which attribute \code{id} equals to \code{i} \\
\hline
\code{querySelectorAll(s)} & All elements matching the CSS selector \code{s} \\
\hline
\code{getElementsByTagName(n)} & All elements of type \code{n} \\
\hline
\code{getElementsByClassName(c)} & All elements which \code{class} attribute contains \code{c} \\
\hline
\end{tabular}
\end{center}
\caption{Standard selectors API}
\label{selectors-api}
\end{figure}

The \code{querySelector} and \code{querySelectorAll} are the most general functions while the others handle special
cases. For the developer it is not convenient to have to master several functions performing similar tasks. In fact,
most JavaScript developers use the jQuery
library~\cite{Bibeault08_jQuery}\footnote{According to \href{http://trends.builtwith.com/javascript}{
http://trends.builtwith.com/javascript}, jQuery is used by more than
40\% of the top million sites} that provides only one high-level function to search for elements.
Listings \ref{vanilla-selectors} and \ref{jquery-selectors} show two equivalent JavaScript programs performing
element searches, the first one using the native APIs and the second one using jQuery.

\begin{figure}
\begin{lstlisting}[language=JavaScript,label=vanilla-selectors,caption=Searching elements in plain JavaScript]
function getWords() {
  var form = document.getElementById('add-user');
  var sections =
    form.getElementsByTagName('fieldset');
  var results = [];
  for (var i = 0 ; i < sections.length ; i++) {
    var words = sections[i]
      .getElementsByClassName('word');
    results[i] = words;
  }
  return results
}
\end{lstlisting}
\end{figure}

\begin{figure}
\begin{lstlisting}[language=JavaScript,label=jquery-selectors,caption=Searching elements in jQuery]
function getWords() {
  var form = $('#add-user');
  var sections = $('fieldset', form);
  return sections.map(function () {
    return $('.word', this)
  })
}
\end{lstlisting}
\end{figure}

jQuery provides an API that is simpler to master because it has less functions, but by doing so it can not benefit
from the performance of the browser’s implementation of specialized search functions (\code{getElementById},
\code{getElementsByTagName} and \code{getElementsByClassName}).

\begin{figure}
\begin{lstlisting}[label=js-scala-selectors,caption=Searching elements in js-scala]
def getWords() = {
  val form = document.find("#add-user")
  val sections = form.findAll("fieldset")
  sections map (_.findAll(".word"))
}
\end{lstlisting}
\end{figure}

Listing~\ref{js-scala-selectors} shows how to implement listing~\ref{jquery-selectors} using js-scala. We
provide two functions for searching elements: \code{find} to find the first element matching a selector and
\code{findAll} to find all the matching elements. During the first evaluation step, these functions try to analyze
the selector that is passed as parameter and, when appropriate, produce code using the specialized  API, otherwise
they produce  code using \code{querySelector} and \code{querySelectorAll}. As a result,
listing~\ref{js-scala-selectors} generates a JavaScript program identical to listing~\ref{vanilla-selectors}: the
high-level abstraction (the \code{find} and \code{findAll} functions) exist only in the initial program, not in the
final JavaScript program.

\begin{figure}
\begin{lstlisting}[label=selector-impl,caption=Selectors optimization]
def find(receiver: Rep[Selector],
         selector: Rep[String]) =
  getConstIdCss(selector) match {
    case Some(id) if receiver == document =>
      DocumentGetElementById(Const(id))
    case _ =>
      SelectorFind(receiver, selector)
  }
\end{lstlisting}
\end{figure}

Listing \ref{selector-impl} shows the implementation of the \code{find} function producing different IR nodes
according to the selector passed as parameter. The \code{getConstIdCss} function analyzes the selector: if it
is a constant \code{String} value containing a CSS ID selector, it returns the value of the identifier. If the
\code{find} function is applied to the \code{document} and to an ID selector, it returns a
\code{DocumentGetElementById} IR node (that is translated to a \code{document.getElementById} call by the code
generator), otherwise it returns a \code{SelectorFind} IR node (that is translated to a \code{querySelector}
call). (FIXME Add a ref to the spec of CSS selectors?)

The same applies to the implementation of \code{findAll}: the selector passed as parameter is analyzed and the
function returns a \code{SelectorGetElementsByClassName} in case of a CSS class name selector, a
\code{SelectorGetElementsByTagName} in case of a CSS tag name selector, and a \code{SelectorFindAll} otherwise.


\section{High-Level Abstractions Shared on Clients and Servers}
\label{contribution-shared}

\subsection{Monads Sequencing}

Null references are a known source of problems in programming languages~\cite{Hoare09_Null,Nanda09_Null}. For
example, consider listing \ref{null-unsafe} finding a particular widget in the page and then a particular
button within the widget. The native \code{querySelector} method returns \code{null} if no node matched the given
selector in the document. If we run this code in a page where the widget is not present, it will throw an error
and stop further JavaScript execution. Defensive code can be written to handle \code{null} references, but leads to
very cumbersome code, as shown in listing \ref{null-defensive}.

\begin{figure}
\begin{lstlisting}[language=JavaScript,label=null-unsafe,caption=Unsafe code]
var loginWidget =
  document.querySelector("div.login");
var loginButton =
  loginWidget.querySelector("button.submit");
loginButton.addEventListener("click", handler);
\end{lstlisting}
\end{figure}


\begin{figure}
\begin{lstlisting}[language=JavaScript,label=null-defensive,caption=Defensive programming to handle null references]
var loginWidget =
  document.querySelector("div.login");
if (loginWidget !== null) {
  var loginButton =
    loginWidget.querySelector("button.submit");
  if (loginButton !== null) {
    loginButton.
      addEventListener("click", handler);
  }
}
\end{lstlisting}
\end{figure}

Some programming languages encode optional values with a monad (\eg \code{Maybe} in Haskell and \code{Option} in
Scala). In that case, sequencing over the monad encodes optional value dereferencing. If the language supports a
convenient syntax for monad sequencing, it brings a convenient syntax for optional value dereferencing, alleviating
developers from the burden of defensive programming.

Listing \ref{null-js-scala} implements in js-scala a program equivalent to listing \ref{null-defensive}. The
\code{for} notation is used to sequence computations over optional values that are encoded with a monad. The
\code{find} function returns a \code{Rep[Option[Element]]} value, that can either be a \code{Rep[Some[Element]]} (if
an element was found) or a \code{Rep[None.type]} (if no element was found). The \code{for} expression contains a
sequence of statements that are executed in order, as long as the previous statement returned a
\code{Rep[Some[Element]]} value.

\begin{figure}
\begin{lstlisting}[label=null-js-scala,caption=Handling null references in js-scala]
for {
  loginWidget <- document.find("div.login")
  loginButton <- loginWidget.find("submit.button")
} loginButton.on(Click)(handler)
\end{lstlisting}
\end{figure}

Such a monadic API brings both safety and expressiveness to developers manipulating optional values but usually
involves the creation of an extra container object holding the optional value. In our case, the monadic API is
used in the initial program but generates code that does not wrap values in container objects but instead checks if
they are \code{null} or not when dereferenced. So the extra container object exists only in the initial program and
is removed during code generation: listing \ref{null-js-scala} produces a code equivalent to listing
\ref{null-defensive}.

\begin{figure}
\begin{lstlisting}[caption=JavaScript code generator for null references handling DSL,label=option-codegen]
override def emitNode(sym: Sym[Any], rhs: Def[Any]) =
  rhs match {
    case OptionIsEmpty(o) =>
      emitValDef(sym, q" $o === null")
    case OptionForeach(o, b) =>
      stream.println(q"if ($o !== null) {")
      emitBlock(b)
      stream.println("}")
    case _ =>
      super.emitNode(sym, rhs)
  }
\end{lstlisting}
\end{figure}

Listing \ref{option-codegen} shows the JavaScript code generator for methods \code{isEmpty} (that checks if the
optional value contains a value) and \code{foreach} (that is called when the \code{for} notation is used, as in
listing \ref{null-js-scala}). The \code{emitNode} method handles \code{OptionIsEmpty} and \code{OptionForeach} nodes
returned by the implementations of \code{isEmpty} and \code{foreach}, respectively. In the case of the
\code{OptionIsEmpty} node, it simply generates an expression testing if the value is \code{null}. In the case of the
\code{OptionForeach} node, it wraps the code block dereferencing the value within a \code{if} checking that the value
is not \code{null}.

Last but not least, we make this abstraction available on server-side by writing a code generator similar to the
JavaScript code generator, but targeting Scala (FIXME Show the code generator?). So the same abstraction is
efficiently translated on both server and client sides.

\subsection{DOM Fragments Definition}

This section shows how we define an abstraction shared between clients and servers, as in the previous section, but
that has different native counterparts on client and server sides. The challenge is to define an API providing a
common vocabulary that generates code using the target platform native APIs.

\begin{figure}
\begin{lstlisting}[language=JavaScript,caption=JavaScript DOM API,label=dom-api]
var articleUi = function (article) {
  var div = document.createElement('div');
  div.setAttribute('class', 'article');
  var span = document.createElement('span');
  var name =
    document.createTextNode(article.name + ': ');
  span.appendChild(name);
  div.appendChild(span);
  var strong = document.createElement('strong');
  var price = document.createTextNode(article.price);
  strong.appendChild(price);
  div.appendChild(strong);
  return div
};
\end{lstlisting}
\end{figure}

\begin{figure}
\begin{lstlisting}[caption=Scala XML API,label=scala-xml-api]
def articleUi(article: Article) =
  <div class="article">
    <span>{ article.name + ": " }</span>
    <strong>{ article.price }</strong>
  </div>
\end{lstlisting}
\end{figure}

A common task in Web applications consists in computing HTML fragments representing a part of the page content. This
task can be performed either from the server-side (to initially respond to a request) or from the client-side (to
update the current page). As an example, listing \ref{dom-api} defines a JavaScript function \code{articleUi} that
builds a DOM tree containing an article description, and listing \ref{scala-xml-api} shows how one could implement
a similar function on server-side using the standard Scala XML library. The reader may notice that the client-side
and server-side APIs are very different and that the client-side API is very low-level and inconvenient to use.

Instead, we provide a common high-level DSL for defining HTML fragments and we make this DSL generate code leveraging
native environments. Listing \ref{forest} shows how to implement our example with our DSL. The \code{el} function
defines an HTML element, eventually containing attributes and children elements. Any children of an element that is
not an element itself is converted into a text node. The children elements of an element can also be obtained
dynamically from a collection, as shown in listing \ref{forest-loops}.

\begin{figure}
\begin{lstlisting}[label=forest,caption=DOM definition DSL]
def articleUi(article: Rep[Article]) =
    el('div, 'class -> 'article)(
        el('span)(article.name + ": "),
        el('strong)(article.price)
    )
\end{lstlisting}
\end{figure}

\begin{figure}
\begin{lstlisting}[label=forest-loops,caption=Using loops]
def articlesUi(articles: Rep[Seq[Article]]) =
    el('ul)(
        for (article <- articles)
        yield el('li)(articleUi(article))
    )
\end{lstlisting}
\end{figure}

The \code{el} function returns an \code{Element} IR node that is a tree composed of other \code{Element} nodes and
\code{Text} nodes. This tree is traversed by the code generators to produce code building an equivalent DOM tree on
client-side and code building an equivalent XML fragment on server-side. When the children of an element are constant
values (as in listing \ref{forest}) rather than dynamically computed (as in listing \ref{forest-loops}), the code
generators inline the loop that adds children to their parent, for better performance. As a result, listing
\ref{forest} generates a code equivalent to listing \ref{dom-api} on client-side and equivalent to
\ref{scala-xml-api} on server-side.

\begin{figure}
\begin{lstlisting}[language=JavaScript,label=js-gen-forest,caption=JavaScript code generator for the DOM fragment
definition DSL]
case Tag(name, children, attrs) =>
  emitValDef(sym, q"document.createElement('$name')")
  for ((n, v) <- attrs) {
    stream.println(q"$sym.setAttribute('$n', $v);")
  }
  children match {
    case Left(children) =>
      for (child <- children) {
        stream.println(q"$sym.appendChild($child);")
      }
    case Right(children) =>
      val x = fresh[Int]
      stream.println(q"for (var $x = 0; $x < $children.length; $x++) {")
      stream.println(q"$sym.appendChild($children[$x]);")
      stream.println("}")
  }
case Text(content) =>
  emitValDef(sym, q"document.createTextNode($content)")
\end{lstlisting}
\end{figure}

\begin{figure}
\begin{lstlisting}[label=scala-gen-forest,caption=Scala code generator for the DOM fragment definition DSL]
case Tag(name, children, attrs) =>
  val attrsFormatted =
    (for ((name, value) <- attrs)
     yield q" $name={ $value }").mkString
  children match {
    case Left(children) =>
      if (children.isEmpty) {
        emitValDef(sym, q"<$name$attrsFormatted />")
      } else {
        emitValDef(sym,
          q"<name$attrsFormatted>{ ${children.map(quote)} }</$name>"
        )
      }
    case Right(children) =>
      emitValDef(sym, q"<$name$attrsFormatted>{ $children }</$name>")
  }
case Text(content) =>
  emitValDef(sym, q"{xml.Text(content)}")
\end{lstlisting}
\end{figure}

Listings \ref{js-gen-forest} and \ref{scala-gen-forest} show the relevant parts of the code generators for this DSL.
They basically follow the same pattern: they visit \code{Tag} and \code{Text} IR nodes and produce the corresponding
elements in the target language. (FIXME More details)

% \subsection{Ad-Hoc Polymorphism}
% 
% Because of the dynamically typed nature of JavaScript, when calling a function there is no proper way to select a
% specialized implementation according to the function’s parameters types. JavaScript is only able to dispatch
% according to a method receiver prototype, \eg{} if one writes \code{foo.bar()} the JavaScript runtime will look into
% the prototype of the \code{foo} object for a property named \code{bar} and will call it. So, the only way to achieve
% \emph{ad hoc} polymorphism on JavaScript objects consists in defining the polymorphic function on the prototypes of
% the objects. However, modifying existing object prototypes is considered bad
% practice~\cite{Zakas12_MaintainableJs}. Another way could consist in manually coding the dispatch logic, by
% registering supported data types at the beginning of the program execution, as described in section 2.4.3
% of~\cite{Abelson83_SICP}, but this solution is painful for developers and incurs a performance overhead.
% 
% We propose to achieve \emph{ad hoc} polymorphism using
% typeclasses~\cite{Wadler89_AdhocPolymorphism,Odersky06_Typeclasses,Oliveira10_Typeclasses} so that it supports
% retroactive extension without modifying objects prototypes. The dispatch logic is type-directed and performed by the
% compiler, so there is no runtime overhead.
% 
% \begin{figure}
% \begin{lstlisting}[label=polymorphism,caption=Ad hoc polymorphism using typeclasses]
% // Interface
% case class Show[A](show: Rep[A => Node])
% 
% // Polymorphic function
% def listWidget[A](items: Rep[List[A]])
%       (implicit A: Show[A]): Rep[Node] =
%   el("ul")(
%     for (item <- items) yield {
%       el("li")(A.show(item))
%     }
%   )
% 
% // Type `User`
% type User = Record {
%   val name: String
%   val age: Int
% }
% // Implementation of Show for a User
% implicit val showUser = Show[User] { user =>
%   el("span", "class"->"user")(
%     user.name + "(" + user.age + " years)"
%   )
% }
% 
% // Main program
% def main(users: Rep[List[User]]) = {
%   document.body.append(listWidget(users))
% }
% \end{lstlisting}
% \end{figure}
% 
% Listing \ref{polymorphism} demonstrates how to define a polymorphic \code{listWidget} function that returns a DOM
% tree containing the representation of a list of items. The \code{Show[A]} typeclass defines how to produce a DOM tree
% for a value of type \code{A}. It is used by the \code{listWidget} function to get the DOM fragments of the list
% items. The listing shows how to reuse the same \code{listWidget} function to show a list of users and a list of
% articles.

\section{Evaluation}
\label{validation}

We measured the runtime performances of the DSL for optional values using a micro-benchmark.

We also have written several implementations of a complete application using different approaches for the client and
server sides, and compared the amount of code written, the runtime performances and the ability to modularize the
code.

Our goal was to evaluate the level of abstraction provided by each solution and their performances. We took the
number of lines of code as a measure of the level of abstraction. We also measured the ability to share code between
client and server sides.

The tests were run on a DELL Latitude E6430 laptop with 8 GB of RAM, on the Google Chrome v27 Web browser.

\subsection{Micro-Benchmark}

We reimplemented the optional value abstraction and a same program using it in the following languages: JavaScript,
HaXe and Java.

\begin{figure}
\centering
\includegraphics[width=8cm]{microbenchmark.png}
\caption{Micro-benchmark on the optional values abstraction}
\label{micro-benchmark}
\end{figure}

Figure \ref{micro-benchmark} shows the micro-benchmark results. Js-scala is between 3 to 100 times faster than
other approaches on our benchmark. Combined with the fact that it provides high-level abstractions, it has a
performance / lines of code ratio more than 4 times higher than other approaches.

\subsection{Real World Application}

Chooze~\footnote{\href{http://chooze.herokuapp.com}{http://chooze.herokuapp.com}} is an existing complete
application for making polls. It allows users to create a poll, define the choice alternatives, share the poll, vote
and look at the results. It contains JavaScript code to handle the dynamic behavior of the application:
double-posting prevention, dynamic form update and rich interaction with the document.

The application was initially written using jQuery. We rewrote it using several technologies for the client-side
part: vanilla JavaScript (low-level code without third-party library), js-scala, GWT and HaXe. In each case we
tried to write the application in an idiomatic way\footnote{Source code is available at
\href{http://github.com/julienrf/chooze}{http://github.com/julienrf/chooze}}.

\subsubsection{Performance}

The benchmark code simulates user actions on a Web page (2000 clicks on buttons, triggering a dynamic update of
the page and involving the use of the optional value monad, the selectors API and the HTML fragment definition API).
Figure \ref{benchmark} shows the benchmark results.

\begin{figure}
\centering
\includegraphics[width=8cm]{chooze.png}
\caption{Benchmarks on a real application}
\label{benchmark}
\end{figure}

The runtime performances of the vanilla JavaScript, HaXe and js-scala versions are similar (though the js-scala
version is slightly slower by 6\%). It is worth noting that the vanilla JavaScript and the HaXe versions use
low-level code compared to js-scala, as shown in the second part of the figure (lines of code): the js-scala version
needs only 74 lines of code while the vanilla JavaScript version needs 116 lines of code (57\% bigger) and the HaXe
version needs 148 lines of code (100\% bigger). The jQuery JavaScript version, of which code is high-level (54 lines
of code, 27\% less than js-scala) runs 10 times slower than the js-scala version.

The last part of the figure compares the runtime performance / lines of code ratio. Js-scala shows the best score,
being 1.48 times better than the vanilla JavaScript version, 1.88 times better than the HaXe version, 3.45 times
better than the GWT version and 7.82 times better than the jQuery JavaScript version.

\subsubsection{Code Reuse}

We were able to share HTML DOM fragment definitions between server-side and client-side in js-scala.

\section{Conclusion}
\label{discussion}

We implemented a high-level language abstracting over client and server heterogeneity but producing efficient code.
Generated code size?

%\appendix
%\section{Appendix Title}
%
%This is the text of the appendix, if you need one.
%
\acks

This work was funded by Zenexity.

\bibliographystyle{abbrvnat}
\bibliography{biblio}
%\begin{thebibliography}{}
%\softraggedright
%
%\bibitem[Smith et~al.(2009)Smith, Jones]{smith02}
%P. Q. Smith, and X. Y. Jones. ...reference text...
%
%\end{thebibliography}
%
\end{document}
