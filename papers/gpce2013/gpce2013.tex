\documentclass[preprint]{sigplanconf}

% The following \documentclass options may be useful:
%
% 10pt          To set in 10-point type instead of 9-point.
% 11pt          To set in 11-point type instead of 9-point.
% authoryear    To obtain author/year citation style instead of numeric.

\usepackage{amsmath}
\usepackage[utf8]{inputenc}
\usepackage[unicode=true,hidelinks]{hyperref}

\usepackage{hyperref}
\usepackage{listings}
%\usepackage[scaled]{luximono}

%\usepackage{natbib}

% ----- begin macros

\lstdefinelanguage{Scala}%
{morekeywords={abstract,%
  case,catch,char,class,%
  def,else,extends,final,for,%
  if,import,implicit,%
  match,module,%
  new,null,%
  object,override,%
  package,private,protected,public,%
  for,public,return,super,%
  this,throw,trait,try,type,%
  val,var,%
  with,%
  yield,%
  lazy%
  },%
  sensitive,%
  morecomment=[l]//,%
  morecomment=[s]{/*}{*/},%
  morestring=[b]",%
  morestring=[b]',%
  showstringspaces=false%
}[keywords,comments,strings]%

\lstdefinelanguage{JavaScript}%
{morekeywords={for, var, attributes, class, classend, do, else, empty, endif, endwhile, fail, function,
functionend, if, implements, in, inherit, inout, not, of, operations, out, return,
then, types, while, use},%
  sensitive,%
  morecomment=[l]//,%
  morecomment=[s]{/*}{*/},%
  morestring=[b]",%
  morestring=[b]',%
  showstringspaces=false%
}[keywords,comments,strings]%

\lstset{language=Scala,%
  mathescape=false,%
%  columns=[c]fixed,%
  aboveskip=\smallskipamount,
  belowskip=\smallskipamount,
%  basewidth={0.5em, 0.4em},%
  basicstyle=\footnotesize,%
  captionpos=b,
  keywordstyle=\bfseries%\sffamily\bfseries,%
%  keywordstyle=\sffamily\bfseries,%
%  xleftmargin=0.5cm
}

\newcommand{\commentstyle}[1]{\slseries{#1}}
\newcommand{\keywordstyle}[1]{\bfseries{#1}}

\lstnewenvironment{slisting}{\lstset{language=Scala}}{}

\newcommand{\code}[1]{\lstinline[language=Scala,columns=fixed,basicstyle=\footnotesize]|#1|}

\def\changemargin#1#2{\list{}{\rightmargin#2\leftmargin#1}\item[]}
\let\endchangemargin=\endlist

%\setlength{\columnseprule}{0.25pt}

%\renewcommand{\note}[1]{$\spadesuit$ \textbf{#1} $\clubsuit$}


%\newcommand{\comment}[1]{}


\newcommand{\ie}{\emph{i.e.}}
\newcommand{\eg}{\emph{e.g.}}
\newcommand{\cf}{\emph{cf.~}}
\newcommand{\etal}{\emph{et al.~}}
\newcommand{\etc}{\emph{etc.}}
\newcommand{\aka}{\emph{a.k.a.}}


\begin{document}

\conferenceinfo{GPCE '13}{October 27--28, 2013, Indianapolis, Indiana, USA} 
\copyrightyear{2013}
\copyrightdata{[to be supplied]} 

\titlebanner{banner above paper title}        % These are ignored unless
\preprintfooter{short description of paper}   % 'preprint' option specified.

\title{Title Text}
\subtitle{You don’t have to trade abstraction for control}

\authorinfo{Julien Richard-Foy\and Olivier Barais\and Jean-Marc Jézéquel}
           {IRISA, Université de Rennes 1}
           {\{firstname\}.\{lastname\}@irisa.fr}

\maketitle

\begin{abstract}
This is the text of the abstract.
\end{abstract}

\category{CR-number}{subcategory}{third-level}

\terms
term1, term2

\keywords
keyword1, keyword2

\section{Introduction}

Web applications are attractive because they require no installation or deployment steps on clients and enable large
scale collaborative experiences. However, writing large Web applications is known to be
difficult~\cite{Mikkonen08_SpaghettiJs,Preciado05_RIAMethodologyNecessity}. One challenge comes from the fact
that the business logic is scattered into heterogeneous client-side and server-side
environments~\cite{Echeverria09_RIA,Kuuskeri09_PartitioningClientServer}. This gives less flexibility in the
engineering process and requires a higher maintainance effort: once you decided to implement a feature on
client-side, changing your *mind* means completely rewriting the feature on server-side (and \emph{vice versa}).
Even worse, logic parts that need to run on both client-side and server-side are duplicated. For instance, HTML
fragments may be built from the server-side when a page is requested by a client, but they may also be built from the
client-side to perform an incremental update subsequent to an user action. How could developers write HTML fragment
definitions once and render them on both client-side and server-side? The more interactive the application is, the
more logic needs to be duplicated between the server-side and the client-side (explain why?).

Using the same programming language on both server-side and client-side can improve the software engineering
process by enabling code reuse between both sides. Incidentally, the JavaScript language -- which is currently the
action language natively supported by most of Web clients -- can be used on server-side, and an increasing
number of programming languages or compiler backends can generate JavaScript code (\eg
Java/GWT~\cite{Chaganti07_GWT}, SharpKit\footnote{\href{http://sharpkit.net}{http://sharpkit.net}},
Dart~\cite{Griffith11_Dart}, Kotlin\footnote{\href{http://kotlin.jetbrains.org/}{http://kotlin.jetbrains.org/}},
ClojureScript~\cite{McGranaghan11_ClojureScript}, Fay\footnote{\href{http://fay-lang.org/}{http://fay-lang.org/}},
Haxe~\cite{Cannasse08_HaXe} or Opa\footnote{\href{http://opalang.org/}{http://opalang.org/}}).

However, this engineering comfort may come at the price of an inefficient runtime: abstracting over platform
differences often means restricting to a subset of common features and losing opportunities to perform platform
specific optimizations. Performance is a primary concern in Web applications, because they are expected to run on a
broad range of devices, from the powerful desktop personal computer to the less powerful smartphone. “Every 100~ms
delay costs 1\% of sales”, said Amazon in 2006. For instance, because the boundaries of the code sent to the client
are less visible when you share code between client-side and server-side, transitive dependencies may pull a lot of
code on the client, causing a high download overhead. Moreover, generating efficient code for heterogeneous platforms
is hard to achieve in an extensible way: the translation of common abstractions like collections into their native
counterpart (JavaScript arrays on client-side and standard library’s collections on server-side) may be hardcoded in
the compiler, but that would not scale to handle all the abstractions a complete application may use (\eg HTML
fragment definitions, form validation rules, or even some business data type that may be represented differently for
performance reasons).

On one hand, for engineering reasons, developers want to write Web applications using a single language, abstracting
over the target platforms differences. But on the other hand, for performance reasons, they want to keep
control on the way their code is compiled to each target platform. How to solve this dilemma?

Compiled domain specific embedded languages~\cite{Elliott2003_Compiling} allow the definition of domain specific
languages (DSLs) as libraries on top of a host language, and to compile them to a target platform. The deep embedding
gives the opportunity to control the code generation scheme for a given abstraction and target platform.

This paper presents such a compiled embedded DSL allowing developers to write Web applications in a single language
which code fragments can be shared between client and server sides, and which is efficiently compiled to each side.
More precisely, we demonstrate the following features:

\begin{itemize}
 \item Type-directed ad-hoc polymorphism on client-side without runtime dynamic dispatch logic;
 \item Usage of monads without extra container object creation;
 \item Ability to define DOM fragments using a common language for server-side and client-side, but that generates
code using standard APIs on both server-side and client-side;
 \item An API for searching in the DOM, that exposes a single entry point but that generates code potentially using
more optimized native APIs.
\end{itemize}

The remainder of this paper is organized as follows. The next section introduces existing approaches for defining
cross-compiling languages. Section \ref{contribution} presents our contribution. Section \ref{validation} evaluates
our contribution. Section \ref{discussion} concludes.

\section{Related Work}

\subsection{Fat Languages}

\subsection{Thin Languages}

\subsection{Deeply Embedded Languages}

Lightweight Modular Staging~\cite{Rompf12_LMSThesis} is a framework for defining deeply embedded DSLs in Scala. It
has been used to define high-performance DSLs for parallel computing~\cite{Brown11_Parallel} and can be used to
generate JavaScript code~\cite{Kossakowski12_JsDESL}.

\section{High-Level Abstractions Generating Efficient (and Heterogeneous) Code}
\label{contribution}

\subsection{Ad-Hoc Polymorphism}

Because of the dynamically typed nature of JavaScript, when calling a function there is no proper way to select a
specialized implementation according to the function’s parameters types. JavaScript is only able to dispatch
according to a method receiver prototype, \eg{} if one writes \code{foo.bar()} the JavaScript runtime will look into
the prototype of the \code{foo} object for a property named \code{bar} and will call it. So, the only way to achieve
\emph{ad hoc} polymorphism on JavaScript objects consists in defining the polymorphic function on the prototypes of
the objects. However, modifying existing object prototypes is considered bad
practice~\cite{Zakas12_MaintainableJs}. Another way could consist in manually coding the dispatch logic, by
registering supported data types at the beginning of the program execution, as described in section 2.4.3
of~\cite{Abelson83_SICP}, but this solution is painful for developers and incurs a performance overhead.

We propose to achieve \emph{ad hoc} polymorphism using
typeclasses~\cite{Wadler89_AdhocPolymorphism,Odersky06_Typeclasses,Oliveira10_Typeclasses} so that it supports
retroactive extension without modifying objects prototypes because it is type-directed: the dispatch happens at
compile-time rather than at runtime.

\begin{figure}
\begin{lstlisting}[label=polymorphism,caption=Ad hoc polymorphism using typeclasses]
// Interface
case class Show[A](show: Rep[A => Node])

// Polymorphic function
def listWidget[A : Show](items: Rep[List[A]]): Rep[Node] = {
  tag("ul")(
    for (item <- items) yield {
      tag("li")(implicitly[Show[A]].show(item))
    }
  )
}

// Type `User`
type User = Record { val name: String; val age: Int }
// Implementation of Show for a User
implicit val showUser = Show[User] { user =>
  tag("span", "class"->"user")(
    text(user.name + "(" + user.age + " years)")
  )
}

// Type `Article`
type Article = Record { val name: String; val price: Double }
// Implementation of Show for an Article
implicit val showArticle = Show[Article] { article =>
  tag("span")(
    text(article.name),
    tag("strong")(text(article.price + " Euros"))
  )
}

// Main program
def main(users: Rep[List[User]], articles: Rep[List[Article]]) = {
  document.body.append(listWidget(users))
  document.body.append(listWidget(articles))
}
\end{lstlisting}
\end{figure}

Listing \ref{polymorphism} demonstrates how to define a polymorphic \code{listWidget} function that returns a DOM
tree containing the representation of a list of items. The \code{Show[A]} typeclass defines how to produce a DOM tree
for a value of type \code{A}. It is used by the \code{listWidget} function to get the DOM fragments of the list
items. The listing shows how to reuse the same \code{listWidget} function to show a list of users and a list of
articles.

\subsection{Monads Sequencing}


The previous work on js-scala \cite{Kossakowski12_JsDESL} showed how to make asynchronous programming more
convenient by making asynchronous calls looking like synchronous calls (\ie{} returning a value instead of taking a
callback as parameter). This work was based on the Scala continuations compiler plugin. We claim that, although this
programming model removes the “callback hell”, it can make the code hard to reason about because there is no explicit
distinction between synchronous and asynchronous computations.

We propose a DSL that explicitly reflects the asynchronous nature of computations in their types and provides methods
turning them into first-class citizens. The DSL is monadic so we can solve the callback hell problem thanks to the
Scala \code{for} notation.

For instance, listing \ref{async-first-class} shows how the listing \ref{async-js} can be re-written using our DSL.
The \code{createVertex} function now returns an asynchronous value instead of taking a callback as
parameter. Then, the \code{for} expression allows us to get the vertex, when available, and to insert it on the user
interface. By making the \code{createVertex} function return an asynchronous value instead of taking a callback as
a parameter, the code is easier to modularize into loosely coupled parts.

\begin{figure}
\centering
\begin{lstlisting}[caption=Asynchronous values are first class citizen,label=async-first-class]
def createVertex(text: Rep[String]): Rep[Future[Vertex]] =
  Ajax.post[Vertex]("/create", new Record { val content = text })

for (vertex <- createVertex("Hello, World!")) {
  addVertex(vertex)
}
\end{lstlisting}
\end{figure}

\begin{figure}
\centering
\begin{lstlisting}[caption=No callback hell,label=async-no-callback]
for {
  foo <- Ajax.get(fooUrl())
  bar <- Ajax.get(barUrl(foo))
  _ <- future(println("bar = " + bar))
  baz <- Ajax.get(bazUrl(bar))
} println(foo + bar + baz)
\end{lstlisting}
\end{figure}

Listing \ref{async-no-callback} translates the callback hell example (listing \ref{async-callback-hell}) using
our DSL. The \code{for} notation can intuitively be thought of as a sequencing notation: whenever the response of the
first Ajax request is available, the next statement is executed, and so on. There is no nested callbacks and the
order of execution is directly reflected by the order of statements.

An asynchronous value of type \code{Rep[A]} is modelled by a value of type \code{Rep[Future[A]]}. Because Scala’s
\code{for} notation is just syntactic sugar for methods \code{foreach}, \code{map} and \code{flatMap}, we are
able to define our DSL by just defining these methods on \code{Rep[Future[A]]} values, with the following semantic:
\begin{itemize}
\item \code{foreach(f: Rep[A] => Rep[Unit]): Rep[Unit]}, eventually does something with the value when available~;
\item \code{map(f: Rep[A] => Rep[B]): Rep[Future[B]]}, eventually transforms the value when available~;
\item \code{flatMap(f: Rep[A] => Rep[Future[B]]): Rep[Future[B]]} also eventually transforms the value when
available~;
\end{itemize}

Asynchronous values can be used and transformed using the \code{foreach}, \code{map} and \code{flatMap} methods,
turning them into first-class
citizens: functions can take them as parameters and return them. Consequently, it is easier to write concurrent code:
listing \ref{async-parallel-1} shows how to run two parallel asynchronous computations and to do something with their
results when both are available. Writing equivalent code in pure JavaScript is tedious because it requires that the
callback passed to each asynchronous computation checks if the other has been completed before, and to store the
value, when availalbe, in a shared place (so both callbacks can get the other’s value).

\begin{figure}
\begin{lstlisting}[caption=Parallel computations in Scala,label=async-parallel-1]
val fooAsync = Ajax.get("/foo")
val barAsync = Ajax.get("/bar")
for {
  foo <- fooAsync
  bar <- barAsync
} println(foo + bar)
\end{lstlisting}
\end{figure}


As an illustration of the staging mechanism, we present a simple DSL to handle null references. This DSL provides an
abstraction at the stage-level that is removed by optimization during the code generation.

Null references are a known source of problems in programming languages~\cite{Hoare09_Null,Nanda09_Null}. For
example, consider the following typical JavaScript code finding a particular widget in the page and a then particular
button in the widget:

\begin{lstlisting}[language=JavaScript,label=null-unsafe,caption=Unsafe code]
var loginWidget = document.querySelector("div.login");
var loginButton = loginWidget.querySelector("button.submit");
loginButton.addEventListener("click", function (e) { ... });
\end{lstlisting}

The native \code{querySelector} method returns \code{null} if no node matched the given selector in the document. If
we run the above code in a page where the widget is not present, it will throw an error and stop further JavaScript
execution. We can write defensive code to handle \code{null} cases, but it leads to very cumbersome code:

\begin{lstlisting}[language=JavaScript,label=null-defensive,caption=Defensive programming to handle null references]
var loginWidget = document.querySelector("div.login");
if (loginWidget !== null) {
  var loginButton = loginWidget.querySelector("button.submit");
  if (loginButton !== null) {
    loginButton.addEventListener("click", function (e) { ... });
  }
}
\end{lstlisting}

We want to define a DSL that has both the safety and performance of listing \ref{null-defensive} but the
expressiveness of listing \ref{null-unsafe}. We can get safety by wrapping potentially null values of type
\code{Rep[A]} in a container of type \code{Rep[Option[A]]} requiring explicit dereferencing, we can get
expressiveness by using the Scala \code{for} notation for dereferencing, and finally we can get performance by
generating code that does not actually wraps values in a container but instead checks if they are \code{null} or not
when dereferenced. The wrapping container exists only at the stage-level and is removed during the code generation.
Here is a Scala listing that uses our DSL (implementation details are given in section \ref{implementation}):

\begin{lstlisting}
for {
  loginWidget <- document.find("div.login")
  loginButton <- loginWidget.find("submit.button")
} loginButton.on(Click) { e => ... }
\end{lstlisting}

The evaluation of the above listing produces a graph of statements from which JavaScript code equivalent to
listing \ref{null-defensive} is generated.

\subsection{DOM Fragments}


In this section we show how we can define a template engine as an embedded DSL with minimal effort. This template
engine is statically typed and able to insert dynamic content in a safe way. It provides a powerful expression
language, requires no extra compilation step and can be used on both client-side and server-side.

Because the template engine is defined as en embedded DSL, we can reuse Scala’s constructs:

\begin{itemize}
\item a function taking some parameters and returning a DOM fragment directly models a template taking parameters and
returning a DOM fragment~;
\item the type system typechecks template definitions and template calls~;
\item the Scala language itself is the expression language~;
\item compiling a template is the same as compiling user code.
\end{itemize}

So the only remaining work consists in defining the DSL vocabulary to define DOM nodes. We provide a \code{tag}
function to define a tag and a \code{text} function to define a text node.

\begin{figure}
\begin{lstlisting}[label=forest-hello,caption=DOM definition DSL]
def vertexDom(v: Rep[Vertex]) =
    tag("g", "class"->"vertex",
             "transform"->("translate("+v.x+","+v.y+")"))(
        tag("rect", "width"->v.width, "height"->v.height)(),
        tag("text", "width"->v.width, "height"->v.height)(
            text(v.content)
        )
    )
\end{lstlisting}
\end{figure}

Listing \ref{forest-hello} uses our DSL and generates a code equivalent to listing \ref{dom-api}. The readability has
been highly improved: nesting tags is just like nesting code blocks, HTML entities are
automatically escaped in text nodes, developers have the full computational power of Scala to inject dynamic data and
DOM fragments definitions are written using functions so they compose just as functions compose. These benefits come
with no performance loss because the DSL generates code building DOM fragments by using the native JavaScript API.

\subsubsection{Reuse the DOM definition DSL on server-side}

Our DSL is equivalent to a template engine with Scala as the expression language. Making it usable on both server and
client sides was surprisingly as simple as defining another code generator for the DSL, producing Scala code.

For instance, the template written in listing \ref{forest-hello} produces the following Scala code usable on
server-side (the generated code for client-side is roughly equivalent to listing \ref{dom-api}):

\begin{lstlisting}
def vertexDom(v: Vertex) = {
  val x0 =
    <text width="{v.width}" height="{v.height}">
      {v.content}
    </text>
  val x1 = <rect width="{v.width}" height="{v.height}" />
  val x2 =
    <g class="vertex" transform="translate({v.x},{v.y})">
      {List(x0, x1)}
    </g>
  x2
}
\end{lstlisting}

We are able to tackle the code sharing issues described in section \ref{assessment} because of the embdedded nature
of our DSLs: dynamic content of templates is written using embedded DSLs too, so their translation into JavaScript
and Scala is managed by their respective code generators.

\subsection{Selectors}

querySelectorAll, getElementsByTagName, getElementsByClassName, getElementById, querySelector

\section{Implementation?}

\section{Evaluation}
\label{validation}

\subsection{Real World Application}

Chooze.

\subsection{Several Implementations}

\subsubsection{Vanilla JavaScript}

\subsubsection{jQuery}

\subsubsection{GWT}

\subsubsection{SharpKit}

\subsubsection{Js-Scala}

\subsection{Benchmarks, Code Metrics}

\section{Conclusion, Future Work}
\label{discussion}

%\appendix
%\section{Appendix Title}
%
%This is the text of the appendix, if you need one.
%
\acks

Acknowledgments, if needed.

\bibliographystyle{abbrvnat}
\bibliography{biblio}
%\begin{thebibliography}{}
%\softraggedright
%
%\bibitem[Smith et~al.(2009)Smith, Jones]{smith02}
%P. Q. Smith, and X. Y. Jones. ...reference text...
%
%\end{thebibliography}
%
\end{document}
